\documentclass{ludis}

% xelatex
\usepackage{fontspec}
\usepackage{xunicode}
\usepackage{xltxtra}

% languages
\usepackage{fixlatvian}
\usepackage{polyglossia}
\setdefaultlanguage{latvian}
\setotherlanguages{english,russian}

% graphics
% \usepackage{pgfplots}
\usepackage{graphicx}
\DeclareGraphicsExtensions{.png,.eps}

% fonts
\setmainfont[Mapping=tex-text]{DejaVu Serif}
\setsansfont[Mapping=tex-text]{DejaVu Sans}
\newfontfamily\russianfont{DejaVu Serif}

% toc
\setcounter{secnumdepth}{3}
\setcounter{tocdepth}{3}

% formatting
% \usepackage{url}
% \usepackage{footnote}
\usepackage{longtable}

% bibtex
\usepackage[pdfauthor={Emīls Šolmanis},%
pdftitle={GPS datu segmentācija},%
pdfkeywords={machine learning, data mining, GPS data analysis, clusterization},%
hidelinks=true,%
pagebackref=false,%
xetex]{hyperref}
\hypersetup{colorlinks=false}
\urlstyle{same}
\usepackage{cite}

%% \usepackage{amsmath}
%% \usepackage{amssymb}
%% \usepackage{enumerate}

\fakultate{Datorikas}
\nosaukums{GPS datu segmentācija}
\darbaveids{Bakalaura}
\autors{Emīls Šolmanis}
\studapl{es09260}
\vaditajs{Asoc. prof., Dr. dat. Jānis Zuters}
\vieta{Rīga}
\gads{2013}

\begin{document}
\maketitle

\begin{abstract-lv}
  Darba mērķis ir izstrādāt neuzraudzītās mašīnmācīšanās algoritmu, kas spētu sadalīt ierakstītu
  GPS datu ceļu atsevišķos fragmentos ar līdzīgām iezīmēm.
fragmentos
  \keywords{mašīnmācīšanās, datizrace, GPS datu analīze, klasterizācija}
\end{abstract-lv}

\begin{abstract-en}
  The goal of this thesis is to develop an unsupervised machine-learning algorithm, that could
  split a recorded path of GPS coordinates into segments based on similar characteristics.
  \keywords{machine learning, data mining, GPS data analysis, clusterization}
\end{abstract-en}

\tableofcontents

\specnodala{Apzīmējumu saraksts}
\setlength\LTleft{0pt}
\setlength\LTright{0pt}
\begin{longtable}{| c | p{28em} |}
  \hline
  \textbf{Apzīmējums} & \textbf{Atšifrējums}\\ 
  \endhead
  \hline
  JVM & Java Virtuālā Mašīna\\
  BLAS & Basic Linear Algebra Subprograms, funkciju kopa, kas nodrošina pamata lineārās algebras 
  funkcionalitāti\\
  LAPACK & Linear Algebra PACKage, BLAS funkciju kopas paplašinājums, balstoties uz to\\
  Paralelizācija & Savstarpēji neatkarīgu programmas daļu vienlaicīga izpilde uz dažādiem procesora
  kodoliem vai procesoriem\\
  \hline
\end{longtable}

\specnodala{Ievads}


\chapter{Uzdevuma izpēte}
\section{Problēmas būtība}


\literatura{es09260}

\end{document}
